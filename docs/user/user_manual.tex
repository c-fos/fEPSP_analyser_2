\documentclass[12pt,a4paper]{report}
\usepackage[utf8]{inputenc}
\usepackage[russian]{babel}
\usepackage[T2A]{fontenc}
\usepackage{amsmath}
\usepackage{amsfonts}
\usepackage{amssymb}
\usepackage{makeidx}
\usepackage{graphicx}
\usepackage[left=2cm,right=2cm,top=2cm,bottom=2cm]{geometry}
%%% позиционирование плавающих объектов%%%
\usepackage[section]{placeins}
%%% Гиперссылки %%%
%\usepackage[linktocpage=true,plainpages=false,pdfpagelabels=false]{hyperref}

\author{Malakhin Ilya}
\title{\huge{FEPSP\_ANALYSER}\\ \large{Пользовательская документация}}
\begin{document}
\maketitle
\chapter{Общее описание программы}
Программа "fEPSP\_analyser" позволяет в полностью автоматизированном режиме рассчитывать амплитуду спайков, присутствующих в записи нейрональной активности, сделанной в программе Ins6.
\chapter{Установка}
\section{Зависимости}
Программа написана на языке Python 2.7 и на данный момент имеет следующие зависимости:
\begin{itemize}
\item СУБД MySQL 5.1.70 (Используемый интерфейс БД, абстрагированный от конкретной реализации БД позволяет работать с любой реляционной БД)
\item GNU R 3.0.1 с установленным модулем nnet любой версии.
\item Ряд модулей Python:
\begin{itemize}
\item PyQt4 4.10.2 - Графический интерфейс
\item matplotlib 1.3.0 - Отрисовка графиков
\item numpy 1.7.1 - Математические операции
\item pywt (PyWavelets) 0.2.2-r1 - вейвлет преобразования  
\item rpy2 2.3.8 - интерфейс из Python к R
\item scipy 0.13.0 - высокоуровневые математические операции
\item sqlalchemy 0.8.2-r1 - Интерфейс в СУБД
\end{itemize}
\end{itemize}
Работа программы возможна и с другими мажорными версиями пакетов, но не гарантируется.
\section{Установка}
\begin{itemize}
\item В любую пользовательскую директорию доступную для записи клонируется master ветка git репозитория с адреса: \textit{http://github.com/c-fos/fEPSP\_analyser\_2}.
\item Устанавливаются перечисленные в зависимостях пакеты
\item Настраиваются MySQL сервер, R, редактируются жестко прописанные пути в файлах rInterface\_lib.py и simple.py
\end{itemize}

\chapter{Настройка}
\section{Настройка MySQL}
После установки MySQL требуется создать схему (по умолчанию fepsp\_db) и пользователя (по умолчанию fepsp\_user с паролем filter123) с полными правами на эту схему. В случае использования других имен, следует отредактировать строку настроек БД в файле dbAccess\_lib\_2.py.
Сервер БД должен быть запущен во время работы fEPSP\_analyser`а.
\section{Настройка R}
После установки R необходимо запустить интерпретатор R и установить пакет nnet выполнив команду install.packages('nnet'). Следует убедится что в файле rInterface\_lib.py прописаны верные пути до директории с библиотеками R.
\section{Настройка}
Открыть интерпретатор Python из директории программы, импортировать модуль dbModel.py ("import dbModel"), выполнить команду ('dbModel.createDB()'). 
\section{Настройка нейронных сетей}
На данный момент пользовательский интерфейс для обучения нейронных сетей не реализован. Обучение выполняется из среды R с использованием скриптов: . Обучающая выборка извлекается из БД и на уровне БД требуется исключить или отредактировать ошибочные данные. После миграции с нейронных сетей реализованных в R на сети созданные в PyBrain переобучение нейронных сетей будет реализовано из пользовательского интерфейса.
\chapter{Расчет}
\section{Требования к данным}
Расчет выполняется над папкой с файлами (.dat и .ins). Для успешного расчета необходимо выполнение следующих требований:
\begin{itemize}
\item Директория с файлами перенесена с компьютера на котором выполнялась запись без потери информации о времени модификации файла. Надо убедится что время изменения файлов соответствует времени создания исходных файлов т.е. времени реальной подачи стимула и записи нейронального ответа. Копирование с Win98 на флеш-карту выполняется с сохранением данной информации. Если при копировании с флеш-карты на компьютер где будет выполнятся расчет это время перезаписывается временем копирования, то для копирования следует использовать команду cp с флагом \textit{--preserve}
\item Названия папки позволяет идентифицировать к какому эксперименту относятся записи содержащиеся в ней
\item Названия файлов позволят идентифицировать к какому этапу эксперимента относится та или иная запись
\end{itemize}
\section{Описание пользовательского интерфейса}
Программа запускается командой `\emph{python fEPSP\_gui.py}` из папки "\emph{путь по которому клонировали репозиторий/fEPSP\_analyser\_new/main}".
После запуска поднимается графический пользовательский интерфейс программы Рис.\ref{fig:analyse}, в терминале нет сообщений об ошибках.

Пользовательский интерфейс представлен одним окном с четырьмя вкладками.
\begin{itemize}
\item Расчет - Вкладка для настройки и запуска анализа над конкретным экспериментом. Риc.\ref{fig:analyse}
\item Удаление - Вкладка позволяющая выбрать один из обработанных экспериментов и удалить его из базы. Рис.\ref{fig:delete}
\item Экспорт - Вкладка для экспорта в формат .csv простого представления обработанных данных (запись-амплитуда спайка). Рис.\ref{fig:export}
\item Настройка - Вкладка для настройки соединения с БД. Рис.\ref{fig:config}
\end{itemize}
\newpage
\subsection{Вкладка "Расчет"}
\begin{figure}[h!]
\vspace{-5pt}
	\center{	\includegraphics[width=14cm,keepaspectratio=true]{img/analyse.png} }\\
 \caption{Пользовательский интерфейс. Вкладка расчет.}
\label{fig:analyse}
\end{figure}

\textbf{Path} - Адресная строка в которой указывается путь до папки с .dat файлами.\\
\textbf{Frequency} - Частота с которой была выполнена запись. Значение по умолчанию. Алгоритм сначала считывает значение частоты из .ins файла, если это не удается, то будет использовано значение по умолчанию.\\
\textbf{Tags} - Поля для перечисление ключевых слов (тэгов), которые будут привязаны к обработанному эксперименту. Теги указывать не обязательно, но они позволяют группировать обработанные эксперименты и могут облегчит идентификацию экспериментов. Тэги перечисляются через запятую. Один тег может быть привязан к любому количеству обработанных экспериментов, также как и эксперимент может иметь произвольное количество тэгов.\\
\textbf{Write to database} - флаг определяющий будут ли записаны данные в БД или в результате анализа будет только создан .png файл с изображением сигнала и найденных спайков. \textit{В настоящий момент в связи с переходом на sqlalchemy флаг работает не корректно и снимать его не рекомендуется.}\\
\textbf{Debug} - Флаг определяющий насколько подробно будут выводится сообщения в терминал. При поднятом флаге в терминал выводятся все сообщения, при опущенном только сообщения об ошибках. Независимо от состояния флага во время расчета формируется файл fEPSP.log в дирректории программы с подробным логом.\\
\textbf{Manual fibre search} - Ручное определение волоконных ответов. При поднятом флаге при обработке каждой записи будет появляться всплывающее окно с изображением сигнала и найденных спайков. В случае если волоконный ответ определился как спайк по нему следует кликнуть правой кнопкой мыши и он будет переопределен как волоконный ответ. \textit{В настоящий момент данный функционал работает не корректно, пофиксю в ближайшее время.}\\
\textbf{Start} - Запуск расчета.\\
\textbf{Smooth coefficient} - коэффициент сглаживания. 7 является оптимальным эмпирически подобранным значением, но в случае если сигнал избыточно сглажен коэффициент можно уменьшить на 1, а если сглажен недостаточно, то увеличить на 1.\\
\subsection{Вкладка "Удаление"} 
\begin{figure}[h!]
\vspace{-5pt}
	\center{	\includegraphics[width=14cm,keepaspectratio=true]{img/delete.png} }\\
 \caption{Пользовательский интерфейс. Вкладка удаление.}
\label{fig:delete}
\end{figure}

\textbf{Refresh} - загрузить список обработанных экспериментов из БД. После загрузки в поле слева от кнопки "Refresh" следует выбрать эксперимент для удаления.
\textbf{Delete saved source data end results} - Удалить из БД эксперимент целиком: метаданные, исходный массив, посчитанные данные.\\
\textbf{Delete results only} - Удалить только посчитанные данные, сохранив методанные (дату, время, теги, названия) и исходные данные (при расчете числовой массив из .dat файла сохраняется в БД). \textit{На данный момент бесполезно, поскольку еще не реализована возможность выполнять расчет с использованием сохраненных в БД исходных данных.}\\
\textbf{Delete} - Удалить выбранный эксперимент.\\
\newpage
\subsection{Вкладка "Экспорт"}
\begin{figure}[h!]
\vspace{-5pt}
	\center{	\includegraphics[width=14cm,keepaspectratio=true]{img/export.png} }\\
 \caption{Пользовательский интерфейс. Вкладка экспорт.}
\label{fig:export}
\end{figure}
\textbf{Refresh} - загрузить список обработанных экспериментов из БД. После загрузки в поле слева от кнопки "Refresh" следует выбрать эксперимент для экспорта.
\textbf{Spike number} - Выбор интересующего спайка. 0 - волоконный ответ, 1 - первый реальный спайк.\\
\textbf{Filename} - Имя текстового файла в который будут записаны данные. Файл будет создан в папке с программой. Название можно писать как с расширением, так и без.\\
\textbf{Export} - Запуск экспорта.\\
\newpage
\subsection{Вкладка "Настройка"}
\begin{figure}[h!]
\vspace{-5pt}
	\center{	\includegraphics[width=14cm,keepaspectratio=true]{img/config.png} }\\
 \caption{Пользовательский интерфейс. Вкладка настройка.}
\label{fig:config}
\end{figure}

\textbf{Server adress} - адрес сервера БД. IP-адрес или localhost, если сервер запущен на локальном компьютере.\\
\textbf{Database name} - Название схемы в базе данных.\\
\textbf{User Name} - Имя пользователя схемы.\\
\textbf{Password} - Пароль пользователя схемы.\\   
\newpage
\section{Запуск расчета}
В адресной строке указываем полный путь до папки с .dat и .ins файлами. Для перехода по дереву каталогов можно пользоваться панелью слева. Для перехода внутрь папки кликаем два раза на папку в левой панели (пока переход с помощью Enter не работает), для перехода на уровень выше есть кнопка справа от адресной строки. Выбрав папку в поле "Tags" указываем через запятую ключевые слова, которые помогут в дальнейшем анализировать посчитанные данные. Тэги можно и не указывать, но они позволяют группировать посчитанные данные и могут облегчить идентификацию экспериментов. При необходимости меняем частоту по умолчанию (частота определяется из файла .ins, но если это не удалось, то используется умолчательная). Нажимаем кнопку "Start". Для повседневного использования флаги переключать не рекомендуется.

\section{Результат расчета}
В результате расчета будут добавлены:
\begin{itemize}
\item В БД записи об обработанной папке (эксперименте), записях и характеристиках найденных спайков.
\item В папку файлы \$ИМЯ\_ФАЙЛА.dat\_graph.png с изображением необработанного сигнала, сглаженного сигнала, найденных спайков. 
\item В папку с программой лог расчета в файле fEPSP.log
\end{itemize}

На данный момент амплитуда спайков записывается не в мВ,а в делениях АЦП (также как и в .dat файлах INS6). Соответственно она может быть переведена в мВ также как это делается в шаблоне EXCEL. 

\chapter{Экспорт}
\section{Настройка и запуск экспорта}
Результат расчета в упрощенном численном виде может быть выгружен из БД на вкладке "Export". Для этого следует нажать кнопку "Refresh", затем в окне слева от кнопки выбрать название интересующего эксперимента (берется с имени обработанной папки), указать амплитуда какого по порядку спайка интересует (0-волоконный ответ, 1-первый реальный спайк), указать полное имя текстового файла (с расширением или без) и нажать кнопку "Export".
\section{Результат экспорта}
В результате будет создан текстовый файл следующей структуры:\\
\begin{tabular}{|p{4.5cm}|p{12cm}|}
\hline 
Название эксперимента & Название эксперимента которому принадлежит запись, название берется с имени папки \\ 
\hline 
Имя файла & Имя обработанной записи без расширения \\ 
\hline 
Номер спайка & Порядковый номер спайка в ответ на первый стимул в проанализированной записи \\ 
\hline 
Время изменения файла & Предполагается что файл изменяется только при первоначальной записи, соответственно это время соответствует времени подачи стамула \\ 
\hline 
Амплитуда & Амплитуда спайка в делениях АЦП (также как в .dat файле) \\ 
\hline 
\end{tabular}
\\ 
Значения колонок разделены запятыми, десятичным разделителем является точка. Т.е. фаил является классическим .csv.
\chapter{Планы}
Планы в порядке приоритета.\\
\begin{tabular}{|c|p{3cm}|p{6cm}|p{4cm}|p{2cm}|}
\hline 
1 & Переход с R на PyBrain, переобучение сетей &  Качественно новый уровень работы с нейронными сетями. & Необходимый этап для дальнейших изменений. & Конец декабря 2013 \\ 
\hline 
2 & Переработка алгоритма сглаживания & Переработка запутанных и вероятно неоптимальных формул, возможно переход на нейронные сети для подбора характеристик фильтров & повышение качества сглаживания, прозрачность алгоритма & январь-февраль 2014 \\ 
\hline 
3 & Алгоритм оценки качества & численная оценка качества сглаживания и поиска стимулов/спайков & необходим для следующего этапа & январь-апрель 2014 \\ 
\hline 
4 & Автоматизация переобучения нейронных сетей & Повышение всех связанных с сетями характеристик анализа & Повышение качества анализа & весна-лето 2014 \\ 
\hline 
\end{tabular} 
\chapter{Здесь мог бы быть FAQ}
По мере поступления вопросов раздел будет пополнятся.
\begin{itemize}
\item Что-то не работает (работает не так), что делать? - Дать мне об этом знать, пока я не буду знать о проблеме я её врятли решу.
\end{itemize}
\chapter{Список сокращений и используемых терминов}
\textbf{БД} - база данных.\\
\textbf{Спайк} - синхронный ответ популяции нейронов на тестирующий стимул зарегистрированный внеклеточно и проявляющийся в изменении электрического потенциала на регистрирующем микроэлектроде.\\
\textbf{Ответ} - один или несколько спайков в ответ на единичный тестирующий стимул.\\
\textbf{Запись} - файл *.dat сформированный программой ins6 и являющийся числовым массивом описывающим изменение электрического потенциала на регистрирующем микроэлектроде от времени.\\
\textbf{Эксперимент} - совокупность записей полученная в рамках одного эксперимента и помещенная в одну директорию.  
 
\end{document}